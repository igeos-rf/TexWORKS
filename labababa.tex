\documentclass[a4paper,12pt]{article}
\usepackage[T2A]{fontenc}
\usepackage[utf8]{inputenc}
\usepackage{cmap}
\usepackage[english, russian]{babel}
\usepackage{wrapfig}
\usepackage[backend=biber,style=gost-numeric]{biblatex} % стиль ГОСТ
\addbibresource{literatura.bib}
\usepackage{misccorr} % в заголовках появляется точка, но при ссылке на них ее нет
\usepackage{amssymb,amsfonts,amsmath,amsthm}  
\usepackage{indentfirst}
\usepackage[usenames,dvipsnames]{color} 
\usepackage[unicode,hidelinks]{hyperref}
% \hypersetup{%
%     pdfborder = {0 0 0}
\usepackage{multirow}
\usepackage{makecell,multirow} 
\usepackage{ulem}
%\usepackage{hyperref}
%\usepackage{graphicx,wrapfig}
%\renewcommand\epsilon{\varepsilon}
% \renewcommand\epsilon{\varepsilon}
%\graphicspath{{img/}}
\usepackage{geometry}
\geometry{left=1cm,right=1cm,top=2cm,bottom=2cm,bindingoffset=0cm,headheight=15pt}
\usepackage{fancyhdr} 
\linespread{1.2} 
\frenchspacing 
\renewcommand{\labelenumii}{\theenumii)} 
% \usepackage{caption}
%%%%%%%%%%%%%%%%%%%%%%%%%%%%%%%%%%%%%%%%%%%%%%%%%%%%%%%%%%%%%%%%%%%%%%%%%%%%%%%
%%%%%%%%%%%%%%%%%%%%%%%%%%%%%%%%%%%%%%%%%%%%%%%%%%%%%%%%%%%%%%%%%%%%%%%%%%%%%%%

\def\labauthor{Сарафанов И.Г.}
\def\labauthors{\labauthor}
\def\labnumber{127}
\def\labtheme{Таблица производных и интегралов}

%%%%%%%%%%%%%%%%%%%%%%%%%%%%%%%%%%%%%%%%%%%%%%%%%%%%%%%%%%%%%%%%%%%%%%%%%%%%%%%
	%применим колонтитул к стилю страницы
\pagestyle{fancy} 
	%очистим "шапку" страницы
\fancyhead{} 
	%слева сверху на четных и справа на нечетных
\fancyhead[L]{\labauthors} 
	%справа сверху на четных и слева на нечетных
\fancyhead[R]{Билеты по механике} 
	%очистим "подвал" страницы
\fancyfoot{} 
	% номер страницы в нижнем колинтуле в центре
\fancyfoot[C]{\thepage} 
%%%%%%%%%%%%%%%%%%%%%%%%%%%%%%%%%%%%%%%%%%%%%%%%%%%%%%%%%%%%%%%%%%%%%%%%%%%%%%%
\usepackage{float}
\usepackage[mode=buildnew]{standalone}
\usepackage{tikz} 
% \usepackage{subcaption}
\usepackage{tikz,csvsimple,physics}
\makeatother
%\usepackage{booktabs}
\usepackage{pgfplots, pgfplotstable}
\usepackage[outline]{contour}
\usepackage{tocloft}
\renewcommand{\cftsecleader}{\cftdotfill{\cftdotsep}}


% вот этот удивительный мир

\begin{document}
\section*{Билет 1. Координатный способ описания движения. Скорость и ускорение}

Движение материальной точки считается заданным, если известно, как меняется её положение со временем в выбранной системе отсчёта. Материальной точкой называют тело, размерами которого в данной задаче можно пренебречь. 
\[
x = x(t), \quad y = y(t), \quad z = z(t).
\]
\[
x = x(t), \quad y = y(t), \quad z = z(t).
\]
В координатном способе описания движения положение точки задаётся её координатами как функциями времени:
\[
x = x(t), \quad y = y(t), \quad z = z(t).
\]
Таким образом, физическая задача сводится к исследованию зависимостей координат от времени.

Скорость характеризует быстроту и направление изменения положения точки. В координатной форме компоненты скорости определяются как производные координат по времени:
\[
v_x = \frac{dx}{dt}, \quad
v_y = \frac{dy}{dt}, \quad
v_z = \frac{dz}{dt}.
\]
Следовательно, производная координаты по времени равна проекции вектора скорости на соответствующую ось. Если данная производная равна нулю, это означает отсутствие движения вдоль выбранной оси, но не обязательно покой точки в целом. Модуль скорости равен
\[
v = \sqrt{v_x^2 + v_y^2 + v_z^2}
\]
и показывает реальную быстроту движения.

Ускорение характеризует изменение скорости со временем. В координатном способе компоненты ускорения равны вторым производным координат по времени:
\[
a_x = \frac{d^2 x}{dt^2}, \quad
a_y = \frac{d^2 y}{dt^2}, \quad
a_z = \frac{d^2 z}{dt^2}.
\]
Физически ускорение может быть связано как с изменением модуля скорости, так и с изменением её направления. В частном случае равномерного прямолинейного движения ускорение равно нулю.

\bigskip

\section*{Билет 2. Векторный способ описания движения. Скорость и ускорение}

Во векторном способе описания движения положение материальной точки задаётся радиус-вектором
\[
\vec r = \vec r(t),
\]
который направлен от начала координат к данной точке. Движение считается заданным, если известна зависимость радиус-вектора от времени. Этот способ описания является более общим и наглядным, так как не привязан к конкретному выбору системы координат.

Скорость в векторной форме определяется как производная радиус-вектора по времени:
\[
\vec v = \frac{d\vec r}{dt}.
\]
Вектор скорости направлен по касательной к траектории движения в данной точке, что связано с тем, что на бесконечно малом участке траектория локально совпадает с касательной.

Ускорение определяется как производная скорости по времени:
\[
\vec a = \frac{d\vec v}{dt} = \frac{d^2 \vec r}{dt^2}.
\]
Ускорение характеризует изменение скорости как по величине, так и по направлению. В общем случае вектор ускорения не совпадает по направлению с вектором скорости. При разложении радиус-вектора по координатным осям векторный способ описания полностью эквивалентен координатному.

\bigskip

\section*{Билет 3. Естественный способ описания движения. Тангенциальное и нормальное ускорения}

\textbf{Естественный способ описания движения материальной точки. 
Скорость и ускорение. Тангенциальное и нормальное ускорения.}

Положение материальной точки в естественном способе описания движения
задаётся длиной пути
\[
l = l(t),
\]
отсчитываемой вдоль траектории от выбранного начала отсчёта.

Скорость материальной точки определяется как производная радиус-вектора
по времени:
\[
\vec v = \frac{d\vec r}{dt}.
\]
В естественном способе описания скорость направлена по касательной
к траектории и может быть записана в виде
\[
\vec v = \frac{dl}{dt}\,\vec\tau = v\,\vec\tau,
\]
где \(v = \dfrac{dl}{dt}\) --- модуль скорости, а \(\vec\tau\) ---
единичный касательный вектор.

Ускорение материальной точки определяется как производная скорости
по времени:
\[
\vec a = \frac{d\vec v}{dt}.
\]
Подставляя выражение для скорости, получаем
\[
\vec a = \frac{d}{dt}(v\vec\tau)
= \frac{dv}{dt}\,\vec\tau + v\,\frac{d\vec\tau}{dt}.
\]

Величина
\[
\vec a_\tau = \frac{dv}{dt}\,\vec\tau
\]
называется тангенциальным ускорением и характеризует изменение модуля
скорости.

Так как модуль касательного вектора постоянен, производная
\(\dfrac{d\vec\tau}{dt}\) направлена перпендикулярно \(\vec\tau\), то есть
вдоль главной нормали \(\vec n\) к траектории.

При движении по окружности радиуса \(R\) при малом перемещении на дугу
\(dl\) касательный вектор поворачивается на угол
\[
d\alpha = \frac{dl}{R}.
\]
Модуль изменения касательного вектора равен \(d\alpha\), поэтому
\[
\frac{d\vec\tau}{dt}
= \frac{d\vec\tau}{dl}\frac{dl}{dt}
= \frac{v}{R}\,\vec n.
\]

Нормальная составляющая ускорения равна
\[
\vec a_n = v\,\frac{d\vec\tau}{dt}
= \frac{v^2}{R}\,\vec n.
\]

Таким образом, ускорение материальной точки разлагается на две
взаимно перпендикулярные составляющие:
\[
\vec a = \vec a_\tau + \vec a_n,
\]
где тангенциальное ускорение отвечает за изменение модуля скорости,
а нормальное ускорение --- за изменение её направления.

\section{Билет 4.}
\textbf{Вращательное движение материальной точки. 
Угловая скорость и угловое ускорение. 
Связь линейных и угловых характеристик движения.}

Вращательным движением материальной точки называется движение точки
по окружности вокруг фиксированной оси. В этом случае положение точки
удобно задавать не декартовыми координатами, а угловой координатой,
характеризующей поворот радиус-вектора точки.

Положение точки при вращательном движении задаётся углом поворота
\[
\varphi = \varphi(t),
\]
отсчитываемым от выбранного начального направления. Угол измеряется
в радианах. Связь угловой координаты с длиной дуги окружности радиуса
\(R\) имеет вид
\[
l = R\varphi.
\]

Угловая скорость характеризует быстроту изменения угла поворота и
определяется как производная угловой координаты по времени:
\[
\omega = \frac{d\varphi}{dt}.
\]
Вектор угловой скорости \(\vec\omega\) направлен вдоль оси вращения,
а его направление определяется правилом правого винта. Для всех точек,
вращающихся вокруг одной и той же оси, вектор угловой скорости одинаков.

Линейная скорость материальной точки связана с угловой скоростью
следующим образом. Дифференцируя выражение \(l = R\varphi\) по времени,
получаем
\[
v = \frac{dl}{dt} = R\frac{d\varphi}{dt} = \omega R.
\]
В векторной форме связь линейной и угловой скоростей записывается как
\[
\vec v = \vec\omega \times \vec r,
\]
где \(\vec r\) --- радиус-вектор точки. Линейная скорость направлена по
касательной к траектории и перпендикулярна радиусу.

Угловое ускорение характеризует скорость изменения угловой скорости и
определяется как
\[
\vec\varepsilon = \frac{d\vec\omega}{dt}.
\]
При вращении вокруг неподвижной оси векторы \(\vec\omega\) и
\(\vec\varepsilon\) направлены вдоль оси вращения.

Ускорение материальной точки при вращательном движении разлагается на
две взаимно перпендикулярные составляющие:
\[
\vec a = \vec a_\tau + \vec a_n,
\]
где \(\vec a_\tau\) --- тангенциальное ускорение, а \(\vec a_n\) ---
нормальное ускорение.

Тангенциальное ускорение связано с изменением модуля линейной скорости:
\[
a_\tau = \frac{dv}{dt}.
\]
С учётом связи \(v = \omega R\) получаем
\[
a_\tau = \varepsilon R.
\]
Тангенциальное ускорение направлено по касательной к траектории и
характеризует изменение величины скорости.
%\includegraphics[]{cir}

Нормальное ускорение обусловлено изменением направления линейной
скорости и направлено к центру окружности. Его модуль равен
\[
a_n = \frac{v^2}{R}.
\]
Подставляя выражение для линейной скорости, получаем
\[
a_n = \omega^2 R.
\]
Нормальное ускорение также называют центростремительным.

При равномерном вращательном движении угловая скорость постоянна,
угловое ускорение равно нулю,
\[
\varepsilon = 0,
\]
и тангенциальное ускорение отсутствует, однако нормальное ускорение
остаётся отличным от нуля, поскольку направление скорости непрерывно
изменяется.

Таким образом, угловые характеристики движения позволяют удобно
описывать вращательное движение материальной точки, а связь линейных и
угловых величин показывает, что изменение модуля скорости определяется
угловым ускорением, тогда как изменение направления скорости связано с
угловой скоростью.

\section*{Билет 5.}
В классической (ньютоновской) механике движение тел описывается относительно системы отсчёта (СО). Одно и то же физическое явление может выглядеть по-разному в разных СО. Чтобы связать описания, используются \textbf{преобразования Галилея} — основа принципа относительности Галилея.

Они применимы между \textbf{инерциальными системами}, движущимися друг относительно друга \textbf{равномерно и поступательно}.

\subsection*{2. Определение систем отсчёта}
Рассмотрим:
\begin{itemize}
    \item Система $K$ — «неподвижная» (лабораторная), инерциальная.
    \item Система $K'$ — движется поступательно относительно $K$ с постоянной скоростью $\vec{V}$.
\end{itemize}
В момент $t = 0$ начала координат совпадают. Время абсолютно: $t' = t$.

Пусть материальная точка имеет:
\begin{itemize}
    \item $\vec{r}(t)$ — радиус-вектор в $K$,
    \item $\vec{r}\,'(t)$ — радиус-вектор в $K'$.
\end{itemize}

\subsection*{3. Преобразование радиус-вектора}
\[
\boxed{\vec{r}(t) = \vec{r}\,'(t) + \vec{V} t}
\]
Это правило сложения перемещений. Пример: лодка в реке — движение относительно воды плюс снос течением.

\subsection*{4. Преобразование скорости}
Дифференцируя по времени:
\[
\boxed{\vec{v} = \vec{v}\,' + \vec{V}}
\]
— классическое правило сложения скоростей.

\subsection*{5. Преобразование ускорения}
Поскольку $\vec{V} = \text{const}$:
\[
\boxed{\vec{a} = \vec{a}\,'}
\]
Ускорение \textbf{инвариантно}. Это обеспечивает инвариантность второго закона Ньютона во всех инерциальных СО.

\end{document}

\end{document}
